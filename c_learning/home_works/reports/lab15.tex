\documentclass[14pt,a4paper]{article}

\usepackage[russian]{babel} % Русский язык
\usepackage{geometry} % Установка полей
\usepackage{graphicx} % Пакет для работы с изображениями
\usepackage{verbatim} % для вставки текста без интерпретации его как LaTeX-команд
\usepackage[hidelinks]{hyperref}

\geometry{top=2cm, bottom=2cm, left=3cm, right=1.5cm}

\begin{document}
% Титульный лист
\begin{titlepage}
\begin{center}
{\large\scshape\bfseries
МИНИСТЕРСТВО НАУКИ И ВЫСШЕГО ОБРАЗОВАНИЯ РОССИЙСКОЙ ФЕДЕРАЦИИ\\
ФЕДЕРАЛЬНОЕ ГОСУДАРСТВЕННОЕ АВТОНОМНОЕ ОБРАЗОВАТЕЛЬНОЕ УЧРЕЖДЕНИЕ ВЫСШЕГО ОБРАЗОВАНИЯ\\
«СЕВЕРО-КАВКАЗСКИЙ ФЕДЕРАЛЬНЫЙ УНИВЕРСИТЕТ»\\
ФАКУЛЬТЕТ МАТЕМАТИКИ И КОМПЬЮТЕРНЫХ НАУК ИМЕНИ ПРОФЕССОРА Н.И.ЧЕРВЯКОВА}
\vfill
{\Large\bfseries ЛАБОРАТОРНАЯ РАБОТА №15}\\[2mm]
{\large Алгоритмизация и программирование}\\[6mm]
{\Large\bfseries Вариант 9}\\[20mm]
\end{center}
\begin{flushright}
\large{
Выполнил студент:\\
Сивко Иван Андреевич\\
студент 2 курса\\
группа ПМИ-б-о-23-2,\\
kаправление подготовки 01.03.02\\[5mm]
Проверил:\\
Ассистент кафедры вычислительной математики и кибернетики, к.ф.-м.н.,\\
Черкашина Анастасия Андреевна}
\end{flushright}
\vfill
\begin{center}
\the\year\ г.
\end{center}
\end{titlepage}

% \tableofcontents
% \newpage

% Основная часть
\begin{center}
    \textbf{Вариант 9}
\end{center}
{\large {\bfseries Цель:} Совершенствование навыков в программировании с использованием\\ указателей.}

\pdfbookmark[1]{Задание 1}{sec1}
\section*{Задание 1}
\textbf{Работа с неструктурированными данными}
\renewcommand{\thesubsection}{\arabic{subsection}} % Задания нумерации для \subsection
\setcounter{subsection}{0} % подпункты с 1
\subsection{Условие}
Для исследования различных методов доступа к файлам данных необходимо выполнить следующие подготовительные действия:

\begin{enumerate}
\item \textbf{Создайте текстовый файл.}

Содержимое файла:
\begin{quote}
    У меня спросили: сколько будет x Опер y?\\
    А я не знаю! А n Опер k? Тоже!\\
    Помогите!
\end{quote}
Например:
\begin{quote}
    У меня спросили: сколько будет 7 * 2?\\
    А я не знаю! А 9 / 4? Тоже!\\
    Помогите!
\end{quote}
Создайте файл с указанным содержимым в текстовом редакторе (например, в Блокноте).

\item \textbf{Обработайте данные.}

Вам известна структура файла. Необходимо:
\begin{itemize}
    \item Вывести содержимое файла на экран.
    \item Записать в выходной файл результаты в формате:
\end{itemize}
\begin{quote}
    x Опер y = Рез1\\
    n Опер k = Рез2
\end{quote}
Например:
\begin{quote}
    7 * 2 = 14\\
    9 / 4 = 2.25
\end{quote}
\item Исходные данные берутся из таблицы согласно варианта:
\begin{figure}[h]
    \centering
    \includegraphics[width=0.8\textwidth]{data/condition15_1.png} % Укажите имя файла изображения
\end{figure}
\end{enumerate}

\subsection{Код:}

\verbatiminput{data/task15_1.c}
\subsection{Результат работы программы}
\begin{figure}[h]
    \centering
    \includegraphics[width=0.8\textwidth]{data/demo15_1.png} % Укажите имя файла изображения
\end{figure}

\pdfbookmark[1]{Задание 2}{sec2}
\section*{Задание 2}
\textbf{Работа со структурированными данными}
\setcounter{subsection}{0}
\subsection{Условие}
Используя полученную при выполнении лабораторной работы 14 программу, реализовать возможность сохранения данных в файл с последующим чтением из файла введенных данных.
\subsection{Код:}
\verbatiminput{data/task15_2.c}
\subsection{Результат работы программы:}
\begin{figure}[h]
    \centering
    \includegraphics[width=0.8\textwidth]{data/demo15_2.png} % Укажите имя файла изображения
\end{figure}
\begin{figure}[h]
    \centering
    \includegraphics[width=0.8\textwidth]{data/demo15_2_2.png} % Укажите имя файла изображения
\end{figure}
\newpage

\section*{Задание 3}
\setcounter{subsection}{0}
\subsection{Условие}
В соответствии с вариантом написать и отладить программу:

Дана информация о пяти больных. Структура имеет вид: фамилия, возраст, пол, давление. Вывести данные о больных с повышенным давлением (более 140) и подсчитать их количество.
\subsection{Код:}
\verbatiminput{data/task15_3.c}
\newpage
\subsection{Результат работы программы:}
\begin{figure}[h]
    \centering
    \includegraphics[width=0.8\textwidth]{data/demo15_3.png} % Укажите имя файла изображения
\end{figure}
\end{document}
